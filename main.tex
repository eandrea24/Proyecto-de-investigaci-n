\documentclass{article}
\usepackage[utf8]{inputenc}
\usepackage[spanish]{babel}
\usepackage{listings}
\usepackage{graphicx}
\graphicspath{ {images/} }
\usepackage{cite}

\begin{document}

\begin{titlepage}
    \begin{center}
        \vspace*{1cm}
            
        \Huge
        \textbf{Proyecto de investigación}
            
        \vspace{0.5cm}
        \LARGE
        Informatica II
            
        \vspace{1.5cm}
            
        \textbf{Erika Andrea Sanchez Salazar}
        \LARGE
         eandrea.sanchez@udea.edu.co
         
        32.256.416
        
       
            
        \vfill
            
        \vspace{0.8cm}
            
        \Large
        Despartamento de Ingeniería Electrónica y Telecomunicaciones\\
        Universidad de Antioquia\\
        Medellín\\
        Septiembre de 2020
            
    \end{center}
\end{titlepage}

\tableofcontents
\newpage
\section{Introducción}\label{intro}
En este documento se recopilaran los tres trabajos de los que esta compuesto el proyecto de investigación de la materia. 

Inicialmente se realizara el taller de memoria, donde se resolverán las preguntas planteadas en el documento suministrado por el docente. Este mismo también es la fuente para resolverlo. Luego se realizaran los dos trabajos faltantes; interrupciones e hilos.

A medida que se avance en el proyecto de investigación se ampliara este documento y sus partes comunes: introducción, indice, referencias. y se agregaran las que se considere pertinente.

\newpage
\section{Taller de Memoria} \label{contenido}

\subsection{Defina que es la memoria del computador.}
la memoria principal del computador es un dispositivo de almacenamiento temporal y alta velocidad de acceso, que se utiliza para poder realizar las actividades mas rapido, ya que si se hiciera con discos duros seria mas lento todo el proceso.

Vamos a citar por ejemplo un artículo de \textbf{Albert Einstein} \cite{einstein}.
También es posible citar libros \cite{dirac} o documentos en línea \cite{knuthwebsite}.\\\\
Revisar en la última sección el formato de las referencias en IEEE.

\subsection{Mencione los tipos de memoria que conoce y haga una pequeña descripción de cada tipo.}
%
A continuación, se presenta el código 
\subsection{Describa la manera como se gestiona la memoria en un computador.}
En la sección \ref{imagenes}, se presentará como añadir ilustraciones al texto.
\subsection{¿Qué hace que una memoria sea más rápida que otra? ¿Por qué esto es importante?}
\newpage
\section{Interrupciones} \label{contenido}
\newpage
\section{Hilos} \label{contenido}
\newpage
\section{Inclusión de imágenes} \label{imagenes}


En la Figura (\ref{fig:cpplogo}), se presenta el logo de C++ contenido en la carpeta images.

\begin{figure}[h]
\includegraphics[width=4cm]{cpplogo.png}
\centering
\caption{Logo de C++}
\label{fig:cpplogo}
\end{figure}

Las secciones (\ref{intro}), (\ref{contenido}) y (\ref{imagenes}) dependen del estilo del documento.

\bibliographystyle{IEEEtran}
\bibliography{references}

\end{document}
